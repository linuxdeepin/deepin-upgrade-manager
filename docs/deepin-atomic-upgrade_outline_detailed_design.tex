\documentclass{utart}

\usepackage{enumitem}
\usepackage{plantuml}
\usepackage{diagbox} 
\usepackage{float}

\title{deepin 原子更新详细设计文档}
\author{liuchenghao}

\setUTClassify{C级商密}

% 设置文档编号
\setUTIndex{UT-YZGX20220224T_SYS026}

% 设置拟制人信息
\setUTFiction{刘成昊}{2022-03-23}

% 设置审核人信息
\setUTReview{闫博文}{2022-04-06}

% 设置批准人信息
\setUTApprove{闫博文}{2022-04-06}

\begin{document}
\utMakeTitle{}{1.0.0}{2022-02-22}
\utMakeChangeLog{
  1.0 & 创建 & 刘成昊 & 2022-03-23 \\
  \hline
  1.1 & 修改技术评审提出问题,详情见详细审计评审报告 & 刘成昊 & 2022-04-06 \\
  \hline
  1.2 & 关键数据结构中增加DBus接口描述 & 刘成昊 & 2022-06-15 \\
  \hline
}
\utMakeTOC

\section{概述}
\subsection{目的}
本文档是针对 deepin 原子更新程序给出的系统详细设计文档,在本文档中,将给出 deepin 原子更新程序的设计原则、静态结构设计、动态流程设计、非功能性设计等内容。
\par
deepin 系统更新程序的设计与实现是基于系统更新的需求分析,总体上将结合结构化设计的方法与文字描述,给出程序结构化的详细设计,与需求分析内容相对应,以保证系统设计的严谨性与可实现性。
在结构化部分,本文档将主要采取 UML 语言的包图、类图、序列图等进行程序设计。
\par
本文档的适用读者为 deepin 原子更新程序的产品经理、设计人员、开发人员、测试人员以及后续维护人员。

\subsection{术语说明}
\begin{itemize}[leftmargin=4em]
\item ostree: 是一个用于对Linux操作系统进行版本更新的系统,它可以被视为 "面向操作系统二进制文件的git" 。通常用来做操作系统项目的持续交付;
\item hardlink: 硬链接,硬链接是通过指向同一个索引节点(inode)来进行文件链接,不支持跨分区或目录链接;
\item snapshot: 快照,是系统在某一时刻状态的副本;
\end{itemize}

\subsection{参考资料}
\begin{itemize}[leftmargin=4em]
\item \href{https://refspecs.linuxfoundation.org/FHS\_3.0/fhs/index.html}{Filesystem Hierarchy Standard.}
\item \href{https://ostree.readthedocs.io/en/latest/}{ostree}
\item \href{https://github.com/linuxdeepin/deepin-styleguide}{deepin 编码规范}
\end{itemize}

\section{系统设计}
deepin 系统更新程序最初的目标是将系统状态的变更操作,如软件的安装、更新或卸载,放入到 chroot 中操作;并提供系统快照的管理功能,如快照生成和快照回滚。
即通过快照构建隔离于当前系统运行环境的 chroot 环境,如此对系统的变更操作,便不会对当前的系统产生影响,仅在变更操作成功后才改变当前系统的状态。

当前阶段,基于时间和 deb 兼容的考虑,系统的变更仍在当前运行环境中进行,不构建 chroot 环境,但在系统变更前生成当前系统的快照。

\subsection{设计原则}
deepin 更新管理程序基于其需求,实现时需遵循以下原则: 
\begin{itemize}[leftmargin=4em]
\item 更新程序应采用管道/过滤器与面向对象相结合的设计风格,流程控制主要采用管道/过滤器风格,功能模块主要采用面向对象风格;
\item 功能模块之间应避免相互调用,降低模块耦合度;
\item 模块应遵循最小化职责原则;
\item 实现时应尽可能只用系统库,禁止引入第三方库,因为需要集成进 \texttt{initrd.img} 中;
\item 执行时应实现失败处理逻辑,尽可能降低对现有系统的影响;
\item 应检查每个流程的产物,确认执行成功;
\item 应遵守 deepin 编码规范;
\end{itemize}

\subsection{子系统设计}
\subsubsection{结构设计}
deepin 更新程序被设计为一个单文件程序,通过命令行参数来执行不同的流程。
其中 init、commit、rollback、clean 等操作,应进行单例检查,不允许并行执行。

整体结构如下: 
\begin{center}
  \begin{adjustbox}{scale=0.75}
    \begin{plantuml}
      @startuml
      package "deepin-atomic-upgrade" {
        package action {
          [init] -[hidden]r-> [list]
          [list] -[hidden]r-> [commit]
          [commit] -[hidden]r-> [rollback]
          [rollback] -[hidden]r-> [clean]

          note top of [init]
          初始化
          end note
          note top of [list]
          版本展示
          end note
          note top of [commit]
          版本提交
          end note
          note top of [rollback]
          版本回滚
          end note
          note top of [clean]
          版本清理
          end note
        }

        package "modules" {
          [upgrader]

          note left of [upgrader]
          更新管理模块,实现初始化、版本查询、
          版本提交、版本回滚等接口
          end note
        }

        package "objects" {
          [init] -d-> [upgrader]
          [list] -d-> [upgrader]
          [commit] -d-> [upgrader]
          [rollback] -d-> [upgrader]
          [clean] -d-> [upgrader]

          [repo] -[hidden]r-> [mountpoint]
          [mountpoint] -[hidden]r-> [mountinfo]
          [mountinfo] -[hidden]r-> [util]

          [upgrader] -d-> [repo]
          [upgrader] -d-> [mountpoint]
          [upgrader] -d-> [mountinfo]
          [upgrader] -d-> [util]

          note bottom of [repo]
          本地 repo 管理模块,
          提供 repo 管理的抽象接口。
          ostree、btrfs、zfs 实现这些接口
          end note
          note bottom of [mountpoint]
          挂载点信息模块,实现挂载点
          的挂载与卸载接口
          end note
          note bottom of [mountinfo]
          /proc/self/mounts 解析模块,
          并实现查询功能
          end note
          note bottom of [util]
          通用功能接口,包括文件复制、
          属主权限修改、字符设备创建
          等功能
          end note
        }
      }
      @enduml
    \end{plantuml}
  \end{adjustbox}

  图 2-1 更新程序结构
\end{center}

\subsubsection{目录组织设计}

deepin 更新程序对系统文件的组成进行了重新设计,系统文件由快照, \texttt{/vendor} ,\texttt{/persistent}组合组成,结构如下: 
\begin{center}
  \begin{adjustbox}{scale=0.75}
    \begin{plantuml}
    @startuml
    package Snapshot {
      [version1] -[hidden]r-> [version2]
      [version2] -[hidden]r-> [versionX]
    }
    
    [Repo] -u-> [version1]
    [Repo] -u-> [version2]
    [Repo] -u-> [versionX]
    
    [ActiveVersion] ..> [versionX]
    [ActiveVersion] -[hidden]r-> [/vendor]
    [OS] <-d- [/persistent]
    [OS] <-d- [/vendor]
    [OS] <-d- [ActiveVersion]
    
    note right of [Repo]
    位于 /persistent/osroot/repo
    end note
    note bottom of Snapshot
    位于 /persistent/osroot/snapshot
    end note
    note right of [/vendor]
    第三方厂商软件安装的目录
    end note
    note right of [/persistent]
    数据分区
    end note
    @enduml
    \end{plantuml}
  \end{adjustbox}

  图 2-2 OS 组成
\end{center}

\subsubsection{repo 设计}
repo 是 deepin 更新程序中最核心的对象,repo 采用了面向接口的设计方法,定义了 repo 必需的接口,并由具体的 repo 管理技术实现。

repo 理论上需要支持 ostree、btrfs、zfs 三种技术,现阶段只支持ostree。

repo 的组成结构如下: 
\begin{center}
  \begin{adjustbox}{scale=0.55}
    \begin{plantuml}
      @startuml
      package Repo {
        [Interface]

        [ostree] -[hidden]r-> [btrfs]
        [btrfs] -[hidden]r-> [zfs]

        [Interface] <-d- [ostree]
        [Interface] <-d- [btrfs]
        [Interface] <-d- [zfs]

        NewRepo - [Interface]
      }
      @enduml
    \end{plantuml}
  \end{adjustbox}

  图 2-3 repo 结构
\end{center}

\subsection{关键流程设计}

\subsubsection{V20 升级至 V23}
\begin{center}
  \begin{adjustbox}{scale=0.65}
    \begin{plantuml}
      @startuml
      actor User
      User -> Upgrade : 执行升级
      Upgrade -> Upgrade : 安装 AtomicManager
      Upgrade -> AtomicManager : 调用接口初始化配置文件,并获取最大分区大小
      alt 空间不足
      AtomicManager --> Upgrade : 最大分区不足存放仓库
      end
      alt 空间不足
      Upgrade --> User : 弹框空间不足仓库构建失败,是否继续升级
      end
      Upgrade -> AtomicManager : 调用接口初始化 repo 仓库
      AtomicManager --> Upgrade : 初始化仓库结果
      alt 初始化失败
      Upgrade --> User : 仓库初始化失败,弹框是否继续升级
      end
      Upgrade -> Upgrade : 执行升级
      Upgrade --> User : 升级结果
      @enduml
      \end{plantuml}
    \end{adjustbox}

    图 2-4 V20升级V23时序图
\end{center}
\begin{itemize}[leftmargin=4em]
  \item AtomicManager 为原子更新程序,Upgrade 为系统更新程序,此图指 V20 升级 V23 的更新程序;
  \item 升级程序需要先调用\texttt{sudo deepin-upgrade-manager --action=prepare}得到需要备份文件路径;
  \item 在初始化配置文件时,需要判断当前最大分区大小,是否能够存放仓库,若空间不足则弹框给用户进行选择是否继续升级;
  \item 在初始化 repo 仓库时,若初始化失败则会进行弹框备份失败,让用户进行选择是否继续升级;  
\end{itemize}

\subsubsection{V23 安装时初始化仓库}
\begin{center}
  \begin{adjustbox}{scale=0.65}
    \begin{plantuml}
      @startuml
      actor User
      User -> DeepinInstaller : 执行安装
      DeepinInstaller -> DeepinInstaller : 释放文件
      DeepinInstaller -> AtomicUpdate : 调用接口初始化仓库
      AtomicUpdate --> DeepinInstaller : 返回仓库初始化结果
      alt 初始化失败
      DeepinInstaller --> User : 仓库初始化失败
      end
      DeepinInstaller --> User : 进入系统
      @enduml
      \end{plantuml}
    \end{adjustbox}

    图 2-5 V23 安装器时序图
\end{center}
\begin{itemize}[leftmargin=4em]
  \item AtomicManager 为原子更新程序,DeepinInstaller 为系统安装器;
  \item 在安装器将 oem 安装包安装完成后需要调用原子更新初始化仓库接口;
  \item 在安装器进行安装时,新增是否需要做初始化备份的选项;
  \item 当更新程序初始化仓库失败,不会影响到安装器主流程;
\end{itemize}

\subsubsection{repo 初始化}
\paragraph{触发场景}
\begin{itemize}[leftmargin=4em]
  \item V20升级V23时会首先初始化此仓库,对文件进行备份;
  \item 安装器安装系统时会提供是否初始化仓库选项,若点击则对仓库进行初始化;
  \item 若仓库未初始化则控制中心提供初始化仓库按钮,用户可以从控制中心进行初始化仓库;
\end{itemize}
\paragraph{逻辑流程}
init 不同场景触发流程一致,其流程如下:
\begin{center}
  \begin{adjustbox}{scale=0.65}
    \begin{plantuml}
      @startuml
      start
      :开始初始化;
      if(配置文件是否存在)then(否)
      :退出;
      stop
      else(是)
      if(repo 是否存在)then(是)
      :退出;
      stop
      else(否)
      :version commit;
      :更新grub模块;
      :系统重启显示启动项;
      stop
      @enduml
    \end{plantuml}
  \end{adjustbox}

  图 2-6 repo init
\end{center}


\subsubsection{rootfs 数据准备}
rootfs 文件夹为系统提交所需内容,其挂载至与 /usr 同级分区中;
\paragraph{触发场景}
\begin{itemize}[leftmargin=4em]
  \item 当 version 提交时会触发此动作;
\end{itemize}
\paragraph{逻辑流程}
\begin{center}
  \begin{adjustbox}{scale=0.65}
    \begin{plantuml}
      @startuml
      start
      :提交时触发数据准备;
      :获取 /usr 文件夹挂载分区;
      if(判断分区空间大小,是否能够存放rootfs)then(是)
      else(否)
      :退出报错;
      stop
      endif
      if(rootfs文件夹是否存在)then(是);
      :删除rootfs文件夹;
      else(否);
      endif
      
      :在/usr 同分区下新建rootfs文件夹;
      :遍历待备份目录文件;
      :rootfs 文件与待备份文件进行对比;
      if(是否在同分区下?)then(是)
      :硬链接;
      else(否)
      :拷贝;
      if(空间是否不足)then(是)
      :删除rootfs文件夹;
      :退出并报错;
      stop;
      else(否)
      endif
      endif
      :rootfs文件准备完成;
      stop;
      @enduml
    \end{plantuml}
  \end{adjustbox}

  图 2-7 rootfs 文件准备完成
\end{center}

\begin{itemize}[leftmargin=4em]
  \item rootfs 由于其内容为系统文件硬链接,所以位置必须与 /usr 分区相同;
  \item ostree 支持软链提交,并在回滚时会保留软链;
\end{itemize}

\subsubsection{version 提交}
\paragraph{触发场景}
\begin{itemize}[leftmargin=4em]
  \item 当 repo 初始化时 sudo deepin-upgrade-manager --action=init,默认会触发一次全
  量提交;
  \item 控制中心提供备份系统接口,手动点击时会触发此动作;
\end{itemize}
\paragraph{逻辑流程}
commit 不同场景触发流程一致,其流程如下: 
\begin{center}
  \begin{adjustbox}{scale=0.65}
    \begin{plantuml}
      @startuml
      start
      :开始提交;
      :解析config.json获取提交目录列表;
      note left: 循环读取config.json,其中 subscribe-list 为需要提交目录
      :rootfs文件准备;
      :生成版本号;
      :提交rootfs文件;
      :显示提交版本号;
      :更新grub;
      :提交完成;
      stop;
      @enduml
    \end{plantuml}
  \end{adjustbox}

  图 2-8 repo commit
\end{center}

\begin{itemize}[leftmargin=4em]
  \item 读取 config.json 配置文件,内容为待备份文件路径,仓库版本,快照路径,临时存储文件路径;
\end{itemize}
\paragraph{错误处理}
\begin{description}[leftmargin=!]
  \item[更新rootfs错误:] : 在每次进行更新时会删除 rootfs 文件夹,再进行重新生成;
\end{description}

\subsubsection{version 回滚}
\paragraph{触发场景}
\begin{itemize}[leftmargin=4em]
  \item 当用户在grub界面选择时会进行触发;
  \item 当控制中心选择回滚时,会进行版本回滚;
\end{itemize}
\paragraph{逻辑流程}
rollback 不同场景触发流程一致,其流程如下: 
\begin{center}
  \begin{adjustbox}{scale=0.75}
    \begin{plantuml}
      @startuml
      start
      :开始回滚;
      if(当前是否在initramfs中)then(是)
      :获取当前挂载根系统的目录;
      else(否)
      endif
      if(确认本地挂载是否与仓库中/etc/fstab挂载点相同)then(不相同)
      :进行重新挂载;
      else(相同)
      endif
      :检出指定版本至<b>快照目录</b>;
      note left: 与 repo 同一分区
      :遍历待回滚目录;
      if(子目录是否与父目录挂载分区相同?)then(是)
      :回滚目录同级创建<b>恢复目录</b>;
      :遍历<b>快照目录</b>文件并与待回滚目录中文件进行比对;
      else(否)
      :子目录下创建<b>恢复目录</b>;
      :遍历<b>快照目录</b>文件并与子目录中文件进行比对;
      endif
      if(是否为相同文件)then(是)
      :硬链接至<b>恢复目录</b>;
      note left: 将待回滚目录文件硬链接至<b>恢复目录</b>
      else(否)
      :拷贝至<b>恢复目录</b>;
      note right: 将<b>快照目录</b>文件拷贝至<b>恢复目录</b>
      endif
      :<b>恢复目录</b>替换待恢复目录;
      note left: 跨分区时,需要将<b>恢复目录</b>文件移至父目录
      :更新grub;
      :回滚完成;
      stop;
      @enduml
    \end{plantuml}
  \end{adjustbox}

  图 2-9 repo version rollback
\end{center}

\begin{itemize}[leftmargin=4em]
  \item 恢复目录: 系统所需的待回滚版本文件,快照目录:  与 repo 同分区目录,内容为仓库检出指定回滚版本硬链接;
  \item 回滚首先会判断当前系统状态,是否在 initramfs 中,若在 initramfs 中就会识别 –rootfs 参数传入的根系统文件路径,再从新的根文件系统路径获取仓库位置;
  \item 回滚时,在同一分区在不同目录下进行挂载: 通过临时文件改名直接将系统目录替换;
  \item 回滚时,在同一分区只在一个目录下进行挂载: 在分区待恢复目录下子目录与父目录同一分区时,需要在恢复目录的同级目录下创建临时目录,当最后文件替换时直接将临时目录进行重命名。当子目录与父目录分区不一致时,需要在子目录下创建临时目录,当最后文件替换时再将临时目录文件进行移出子目录下;
  \item 在数据替换完成之后,再将系统的 grub 进行更新,显示当前系统的 grub 引导;
\end{itemize}

\paragraph{错误处理}
\begin{description}[leftmargin=!]
  \item[根路径未获取到:] 退出程序并进行报错处理;
  \item[在 initramfs 回滚时异常:] 在临时目录替换待恢复目录时,将 /boot 目录设置为最后替换,防止失败时无法找到引导项;
\end{description}

\subsubsection{version 清理}
\paragraph{触发场景}
\begin{itemize}[leftmargin=4em]
  \item 当控制中心选择清理时,会进行版本清理;
\end{itemize}

\paragraph{逻辑流程}
clean 不同场景触发流程一致,其流程如下: 
\begin{center}
  \begin{adjustbox}{scale=0.65}
    \begin{plantuml}
      @startuml
      :传入 clean 参数调用更新程序;
      if (是否已有更新程序运行) then (Y)
      :显示错误信息;
      else
      :列出所有 version;
      if (version 个数是否大于最大保留数量) then (N)
      :显示提示信息;
      else
      :生成待清理版本列表;
      note right:初始版本不显示
      while (遍历待清理版本列表)
      :获取版本对应的提交列表;
      :删除版本;
      while (遍历提交列表)
      :删除提交;
      endwhile
      endwhile
      endif
      endif
      stop
      @enduml
    \end{plantuml}
  \end{adjustbox}

  图 2-10 repo version clean
\end{center}
\begin{itemize}[leftmargin=4em]
  \item 版本清理时需要将 ostree 的 branch 中 commit 进行全部删除;
  \item 最大保留数量在控制中心可以设置;
  \item 生成待清理版本时不显示初始版本,防止无法还原初始系统;
\end{itemize}

\subsubsection{grub 更新}
\paragraph{触发场景}
\begin{itemize}[leftmargin=4em]
  \item 当系统进行版本提交时会触发,显示最新版本;
  \item 当系统进行版本清理会触发,已清理的版本不会进行显示;
  \item 当系统进行版本回滚时触发,显示回滚后的版本;
\end{itemize}
\paragraph{逻辑流程}
update grub 不同场景触发流程一致,其流程如下: 
\begin{center}
  \begin{adjustbox}{scale=0.65}
    \begin{plantuml}
      @startuml
      :开始更新引导项;
      :设置环境变量;
      note right: 限制grub显示版本数量
      :获取当前已备份版本;
      if(当前/boot中内核与已备份内核是否相同)then(是)
      :硬链接本机内核至/boot/snapshot 文件夹下;
      note left:cp -l ”/boot/*” ”/boot/snapshot”
      else(否)
      :将已备份版本内核文件拷贝至/boot下;
      note right:cp ”/persistent/osroot/snapshot/V23.*/boot/*” ”/boot/snapshot”
      endif
      :设置init启动参数;
      :写入grub.cfg配置文件中;
      note right: /boot/grub/grub.cfg
      stop
      @enduml      
    \end{plantuml}
  \end{adjustbox}

  图 2-11 update grub
\end{center}
\begin{itemize}[leftmargin=4em]
  \item 引导项显示回滚最大个数可以通过环境变量进行设置;
  \item 回滚时是将脚本放入 \texttt{grub中}再利用 \texttt{update-grub}命令进行更新;
  \item 当内核拷贝时,若系统内核与已备份内核 hash 相同则直接将系统内核硬链接,从而节约空间;
\end{itemize}


\subsection{数据结构设计}
\subsubsection{版本号}
\paragraph{版本号格式}
本地仓库的版本号由 \texttt{<distribution>.<major>.<minor>.<date>} 组成。
\texttt{distribution} 是系统大版本,只允许包含字母和数字,字母全小写,与 \texttt{debian} 中的含义保持一致;
\texttt{major} 和 \texttt{minor} 是无符号整数,\texttt{major} 在 \texttt{date} 发生变化时自增,并将 \texttt{minor} 重置;
\texttt{minor} 则是在 \texttt{date} 不变时自增, \texttt{major} 则保持不变;
\texttt{date} 是当前的时间,由 \texttt{YearMonthDay}组成,如 \texttt{20220222} 。

这里用下面的例子说明版本号的变更方式: 

\begin{quote}
假设 \texttt{distribution} 为 \texttt{V23} ,\texttt{date} 为 \texttt{20220222} 。

则初始化本地仓库时,版本号则为 \texttt{V23.0.0.20220222} ,在日期不变时,对系统做了一些修改,再次提交时,版本则变为 \texttt{V23.0.1.20220222} 。

而在一天后,再次提交时,版本号则为 \texttt{V23.1.0.20220223}。
\end{quote}
\paragraph{版本号比较逻辑}
\begin{quote}
Example 1 : 假设 distribution 为 V23,在 2022 年 4 月 1 日安装好系统进行了两
次提交,则两次的版本号分别为 V23.0.1.20220401 和 V23.0.2.20220401 此时则可
比较 minor 的大小。\\

Example 2 : 假设 distribution 为 V23,在 2022 年 4 月 1 日安装好系统进行了一
次提交,在 2022 年 4 月 5 日安装好系统进行了一次提交,则两次的版本号分别为
V23.0.0.20220401 和 V23.1.0.20220405 此刻可以比较 major 大小。
\end{quote}
\subsubsection{Repo}
Repo 中定义了仓库管理的接口,OSTree 进行实现,数据结构如下: 
\begin{center}
  \begin{adjustbox}{scale=0.75}
    \begin{plantuml}
      @startuml
      package Repo {
        interface Repository {
          Init() error
          Exist(string version)(bool)
          Last() (string, error)
          List() ([]string, error)
          Snapshot(version, snapDir string) error
          Commit(version, subject, dataDir string) error
          Diff(baseVersion, targetVersion, dstFile string) error
          Cat(version, filepath, dstFile string) error
          Previous(version string) (string, error)
        }
        class OSTree implements Repository {
          - repoDir string
        }
      }
      @enduml
    \end{plantuml}
  \end{adjustbox}

  图 2-12 repo object
\end{center}
\begin{itemize}[leftmargin=4em]
  \item Init : 初始化仓库,并进行一次全量提交;
  \item Exist : 判断仓库是否存在此版本;
  \begin{itemize}[leftmargin=4em]
    \item version : 字符串类型,传入需判断版本号;
    \item return : bool 表示此版本是否存在,若存在则为 true ,不存在则为 false;
  \end{itemize}  
  \item Last : 获取最后一次提交时版本;
  \begin{itemize}[leftmargin=4em]
    \item return : 字符串类型,返回最后一次提交版本;
  \end{itemize} 
  \item List: 获取当前仓库所有版本号;
  \begin{itemize}[leftmargin=4em]
    \item return : 字符串数组类型,返回的当前仓库版本号列表;
  \end{itemize} 
  \item Snapshot: 获取指定版本快照;
  \begin{itemize}[leftmargin=4em]
    \item version : 字符串类型,指定版本号;
    \item snapDir : 字符串类型,获取快照时,存放本地路径;
  \end{itemize} 
  \item Commit: 提交文件至仓库;
  \begin{itemize}[leftmargin=4em]
    \item version : 字符串类型,指定版本号;
    \item subject : 字符串类型,此次提交的主题;
    \item dataDir : 此次提交文件路径;
  \end{itemize}   
  \item Diff: 比较两次提交,并输出不同;
  \begin{itemize}[leftmargin=4em]
    \item baseVersion : 字符串类型,待比较版本号;
    \item targetVersion : 字符串类型,需比较版本号;
    \item dstFile : 字符串类型,两版本中不同内容;
  \end{itemize} 
  \item Cat: 获取指定版本,指定内容;
  \begin{itemize}[leftmargin=4em]
    \item version : 字符串类型,指定版本号;
    \item ilepath : 字符串类型,指定版本文件中文件路径;
    \item dstFile : 字符串类型,获取指定分支指定文件内容;
  \end{itemize} 
  \item Previous: 获取指定版本上一个版本号;
  \begin{itemize}[leftmargin=4em]
    \item version : 字符串类型,指定版本号;
    \item return : 字符串类型,返回上一个版本号;
  \end{itemize} 
\end{itemize}

\subsubsection{DBus}
DBus 中定义了外部调用的接口。 
\begin{itemize}[leftmargin=4em]
  \item service name:org.deepin.AtomicUpgrade1
  \item object path:/org/deepin/AtomicUpgrade1
  \item interface name:org.deepin.AtomicUpgrade1
\end{itemize}

\paragraph{Methods}
\begin{itemize}[leftmargin=4em]
  \item Commit (IN String  subject): 提交当前系统至本地仓库,异步调用,立即返回,若已存在提交或回滚进程则返回dbus错误;
  \item Rollback (IN String  version): 回滚系统至指定版本,异步调用,立即返回,若已存在提交或回滚进程则返回dbus错误;
  \item Delete (IN String  version): 删除仓库指定版本,当前激活版本与初始版本无法删除;
  \item ListVersion (OUT Array<String> list): 获取当前仓库的版本列表;
  \item QuerySubject (IN Array<String> versions)(OUT Array<String> list): 获取指定版本列表对应主题信息;
\end{itemize} 

\paragraph{Properties}
\begin{itemize}[leftmargin=4em]
  \item ActiveVersion: 当前系统中正在使用仓库的系统版本;
\end{itemize}

\paragraph{Signals}
\begin{itemize}[leftmargin=4em]
  \item StateChanged (Int32 operate, Int32 state, String version,String message): 当前程序改变状态信号通知;
\end{itemize}

\subsubsection{config.json}
config.json结构中包含生成,提交,回滚操作所需要的基本数据,数据格式为json,存放路径为\texttt{/persistent/osroot/config.json}。

\begin{itemize}[leftmargin=4em]
  \item version: 当前原子更新版本;
  \item distribution: 当前系统发行版本;
  \item cache\_dir: 当前释放系统文件缓存地址,一般会挂载至根分区;
  \item active\_version: 当前系统使用仓库文件的激活版本;
  \item max\_repo\_retention: 最大保存版本数量;
  \item repo\_list: 数组存储 repo 内部信息;
    \begin{itemize}[leftmargin=4em]
    \item repo: repo仓库路径;
    \item snapshot\_dir: 快照存放路径;
    \item subscribe\_list: 需要备份文件路径;
    \end{itemize}
\end{itemize}

\section{非功能性设计}
\subsection{性能}
\subsubsection{性能测试}
本次测试内容为初始化 repo ,二次提交三次提交仓库,在 HDD 和 SSD 下,在跨分区(跨分区: 将/boot,/var,/ur/local挂载至其他分区)和非跨分区的不同情况下表现,其测试结果如下:
\begin{table}[H]
  \begin{tabular}{|c|ccc|}
  \hline
             & \multicolumn{3}{c|}{i9-10885H/8GB/4核/SSD/虚拟机(全盘)}                          \\ \hline
             & \multicolumn{1}{c|}{首次(包含初始化repo)} & \multicolumn{1}{c|}{二次(重启后)} & 三次(重启) \\ \hline
  备份时间       & \multicolumn{1}{c|}{96}            & \multicolumn{1}{c|}{86}      & 88     \\ \hline
  准备rootfs时间 & \multicolumn{1}{c|}{11}            & \multicolumn{1}{c|}{9}       & 11     \\ \hline
  otree提交时间  & \multicolumn{1}{c|}{82}            & \multicolumn{1}{c|}{76}      & 74     \\ \hline
  删除缓存时间     & \multicolumn{1}{c|}{3}             & \multicolumn{1}{c|}{3}       & 3      \\ \hline
  \end{tabular}
  \end{table}
\begin{table}[H]
  \begin{tabular}{|c|ccc|}
  \hline
              & \multicolumn{3}{c|}{i7-10700/8GB/4核/HHD/虚拟机(全盘)}                           \\ \hline
              & \multicolumn{1}{c|}{首次(包含初始化repo)} & \multicolumn{1}{c|}{二次(重启后)} & 三次(重启) \\ \hline
  备份时间       & \multicolumn{1}{c|}{379}           & \multicolumn{1}{c|}{296}     & 299    \\ \hline
  准备rootfs时间 & \multicolumn{1}{c|}{75}            & \multicolumn{1}{c|}{75}      & 82     \\ \hline
  otree提交时间  & \multicolumn{1}{c|}{294}           & \multicolumn{1}{c|}{211}     & 210    \\ \hline
  删除缓存时间     & \multicolumn{1}{c|}{9}             & \multicolumn{1}{c|}{7}       & 7      \\ \hline
  \end{tabular}
  \end{table}
\begin{table}[H]
  \begin{tabular}{|c|ccc|}
  \hline
              & \multicolumn{3}{c|}{i9-10885H/8GB/4核/SSD/虚拟机(跨分区)}                         \\ \hline
              & \multicolumn{1}{c|}{首次(包含初始化repo)} & \multicolumn{1}{c|}{二次(重启后)} & 三次(重启) \\ \hline
  备份时间       & \multicolumn{1}{c|}{115}           & \multicolumn{1}{c|}{109}     & 107    \\ \hline
  准备rootfs时间 & \multicolumn{1}{c|}{34}            & \multicolumn{1}{c|}{38}      & 38     \\ \hline
  otree提交时间  & \multicolumn{1}{c|}{78}            & \multicolumn{1}{c|}{67}      & 65     \\ \hline
  删除缓存时间     & \multicolumn{1}{c|}{3}             & \multicolumn{1}{c|}{4}       & 3      \\ \hline
  \end{tabular}
  \end{table}
本次测试内容为在 HDD 和 SSD 不同情况下解压 systemfiles.squashfs 下表现,其测试结果如下:
\begin{table}[H]
  \begin{tabular}{|c|cc|cc|}
  \hline
   & \multicolumn{2}{c|}{i9-10885H/8GB/4核/SSD/虚拟机} & \multicolumn{2}{c|}{i7-10700/8GB/4核/HHD/虚拟机} \\ \hline
                          & \multicolumn{1}{c|}{第一次} & 第二次 & \multicolumn{1}{c|}{第一次} & 第二次 \\ \hline
  repo预制作 & \multicolumn{1}{c|}{42}  & 43  & \multicolumn{1}{c|}{280} & 310 \\ \hline
  无repo   & \multicolumn{1}{c|}{43}  & 41  & \multicolumn{1}{c|}{140} & 121 \\ \hline
  \end{tabular}
  \end{table}
\begin{table}[H]
  \begin{tabular}{|c|cc|}
  \hline
                          & \multicolumn{2}{c|}{i7-10700/8GB/4核/HHD/物理机} \\ \hline
                          & \multicolumn{1}{c|}{第一次}        & 第二次        \\ \hline
  repo预制作 & \multicolumn{1}{c|}{118}        & 122        \\ \hline
  无repo   & \multicolumn{1}{c|}{65}         & 86         \\ \hline
  \end{tabular}
  \end{table}
  
\subsubsection{性能分析}
\begin{itemize}[leftmargin=4em]
\item 4 核/SSD(增加 1 分半左右,时间增长 70\% 左右),4 核心/HDD(增加 5 分钟+,时间增长 55\% 左右);
\item 备份时间主要受 ostree 提交操作影响,占总耗时 70\% 以上;
\item 在安装阶段生成备份时,预制作 repo 在 SSD 场景有略微(10\% 左右)优化,在 HDD性能会退化;
\item 准备备份集合 rootfs 时间,在 SSD 上占比 10\% 左右,在 HDD 上 占比 20\%+。(潜在优化项);
\end{itemize}

\subsubsection{性能劣化消减方案}
\begin{itemize}[leftmargin=4em]
\item 安装器中添加是否生成初始化备份的选项,默认不勾选,减少默认状态下对安装时长的影响;
\item 数据回滚时,在待备份目录下创建临时目录,再将回滚目录与待备份目录进行比对,若相同则直接将待备份目录文件硬链接至临时目录下,从而减少回滚拷贝文件时间;
\end{itemize}

\subsection{安全性}
deepin 更新程序在数据安全,权限安全进行设计:

\subsubsection{数据安全}
\begin{itemize}[leftmargin=4em]
  \item 由于程序会访问修改系统文件,所以在更新程序中发生的任何操作都由管理员操作执行;
\end{itemize}

\subsubsection{权限安全}
\begin{itemize}[leftmargin=4em]
  \item 在程序提交和回滚系统文件时不会改变其权限,所以不会导致系统文件的权限问题;
\end{itemize}

\subsubsection{参数安全}
\begin{itemize}[leftmargin=4em]
  \item 在程序内部对恶意传参和参数错误传入都会进行比对,若程序内无此参数设置则会进行报错退出;
\end{itemize}

\subsection{可靠性}
\subsubsection{系统版本提交}
\begin{quote}
出现系统断电:
\begin{itemize}
  \item 利用 ostree 的提交功能,在未完全提交时中断并不会显示至提交列表中;
\end{itemize}
出现系统磁盘不足:
\begin{itemize}
  \item 更新程序的仓库存放至最大分区中,若其在根分区并空间不足时,会进行报错并退出;
  \item 在文件提交时会新建 rootfs 文件夹至与 /usr 同分区下,其内容大部分为硬链接,若空间不足时,会退出程序报错并将 rootfs 文件删除;
\end{itemize}
出现数据磁盘不足:
\begin{itemize}
  \item  当版本提交时,若数据磁盘空间不足,会进行报错退出并将新建 rootfs 文件进行删除;
\end{itemize}
\end{quote}

\subsubsection{系统版本回滚}
\begin{quote}
出现系统断电:
\begin{itemize}
  \item 每次仓库检出版本会将上次缓存进行删除,再重新 checkout;
  \item 在版本回滚文件替换时,会将引导文件最后替换防止由于系统断电,导致引导错误无法启动系统;
\end{itemize}
出现系统磁盘不足:
\begin{itemize}
  \item 更新程序的仓库存放至数据分区,会保证其足够大小,仓库回滚时首先在 /usr 同分区下新建 rootfs 文件夹用于存放回滚版本内容,若分区空间不足则会进行报错退出并将其删除;
\end{itemize}
\end{quote}
\subsection{兼容性}

  \subsubsection{V20的升级与回滚}
    \begin{itemize}
      \item 在 V20 升级 V23 时,若数据分区的大小不足无法存放仓库,则会自动选择最大分区进行存储;
      \item 在 V20 升级 V23 时,会将 init 程序进行升级,需其支持参数传入指定内核进行系统启动,防止系统回滚失效;
      \item V20 与 V23 交互详细流程在本文 2.3.2 详细说明,此功能为原子更新主要功能点;
    \end{itemize}
  \subsubsection{V20 兼容性问题}
  目前 V20 的运行兼容只考虑在 V20 与 V23 的升级回滚,暂不考虑 V20 本身的升级回滚,其不兼容问题如下;
  \begin{quote}
    规避出现运行系统环境不兼容问题:
    \begin{itemize}
      \item 更新程序为 go 开发,静态编译,运行时不依赖系统环境;
    \end{itemize}
    规避出现运行系统目录不兼容问题:
    \begin{itemize}
      \item 在 repo 构建时,只与目录相关联,系统的目录结构并未变更;
    \end{itemize}
    \end{quote}

\section{部署与实施}
\subsection{ISO 预装}
    在 V23 的 ISO 镜像中,需要把原子更新模块期望的系统目录进行集成,其在本文的 2.2.4 和 2.3.1 有详细说明;
\subsection{安装器初始化}
    在 安装器对 ISO 安装时,需要对已在ISO集成的期望目录进行循环解压,其在本文 2.3.3 有详细交互图说明;

\end{document}
